\documentclass{beamer}

\usepackage[utf8]{inputenc}   % Підтримка UTF-8
\usepackage[ukrainian]{babel} % Підтримка української мови
\usepackage[ukrainian=nohyphenation]{hyphsubst}
\usepackage{booktabs}
\usepackage[T2A]{fontenc}      % Кодова таблиця для кирилиці
\usepackage{amsmath, amsfonts} % Для математики, якщо потрібно
\usepackage{hyperref}          % Для створення посилань
\usepackage{listings}          % Пакет для вставки коду
\usepackage{graphicx}
\usepackage{csvsimple}
\usepackage{parskip}
\usepackage{csquotes}
\usepackage{xcolor}
\usepackage{multicol} % Для багатостовпчикового тексту

\usetheme{Madrid}

% Прибираєм навігацію з кожного слайду
\beamertemplatenavigationsymbolsempty

\title{Лабораторна робота №2}
\subtitle{Статистичне виведення}
\subtitle{Команда №9}

% [], щоб прибрати імена з кожного слайду
\author[]{
  Баранівська В.О.,
  Корсун Є. В.,
  Хмарук О. Ю.,
  Літковський А.С.,
  Кудін Н. А.
}
\date{2025}

\begin{document}

\frame{\titlepage}

% \graphicspath{{../../../}} % Ensure this path is correct or remove it if not needed

%Короткий підсумок ЛР 1 (якщо є свіжі погляди, можна ще зменшити/додати/змінити)
\begin{frame}
  \section{Узагальнення ЛР 1}

  \frametitle{Зміст}
  \tableofcontents[currentsection]
\end{frame}

\begin{frame}
  \frametitle{Набір даних}

  Було вирішено дослідити якість повітря Тайваню. Уряд провінції намагається
  контролювати та покращувати якість повітря. Тому 17 грудня 2017 року була введена
  реформа \textit{Air Pollution Control Action Plan}.

  \begin{center}
    
  \end{center}
\end{frame}

\begin{frame}
  \frametitle{Набір даних}

  Загальний опис датасету:

  \begin{enumerate}
    \item Кількість рядків: 5\,882\,208
    \item Кількість стовпців: 25

    \begin{itemize}
      \item Числові: 19
      \item Факторні: 4
      \item Дата: 1
      \item Інші: 1
    \end{itemize}
  \end{enumerate}
\end{frame}

\begin{frame}
  \frametitle{Питання EDA}

  \begin{enumerate}
    \item Чи впливає швидкість вітру на концентрацію частинок PM2.5 і PM10?
    \item Чи існує кореляція між рівнем забруднення повітря і типом головного забруднювача
    в різних районах?
    \item Як зміни в концентрації озону  впливають на загальний рівень забруднення повітря?
    \item Які регіони мають найвищий середній рівень забруднення повітря протягом року?
    \item Як змінюється якість повітря протягом доби в різних районах?
  \end{enumerate}
\end{frame}

\begin{frame}
  \frametitle{Питання EDA}

  \begin{enumerate}
    \setcounter{enumi}{5}

    \item Як змінюється концентрація PM2.5 і PM10 в залежності від швидкості вітру і напрямку вітру
    в різних регіонах?
    \item Як змінився загальний рівень забруднення по регіонам після початку реформи?
    \item Чи існує залежність між початком реформ та показниками забруднення?
    \item Як змінюється якість повітря залежно від станції виміру у містах?
  \end{enumerate}
\end{frame}

\begin{frame}
  \section{Висновки EDA}

  \frametitle{Зміст}
  \tableofcontents[currentsection]
\end{frame}

\begin{frame}
  \frametitle{Висновки EDA}

Було проведено розвідковий аналіз результатів виміру якости повітря у регіонах Тайваню з 2016 року по 2024. 

Не вдалось відповісти на деякі поставлені питання, 
через значну кількість відсутніх даних у стовпці $'pollutant'$, а саме: 
\begin{enumerate}
    
    \item  Чи існує кореляція між рівнем забруднення повітря (AQI) і типом головного забруднювача (pollutant) в різних районах?
    
\end{enumerate}
Відповідь на це питання сподіваємось уряд Тайваню, зможе в скорому часі оприлюднити, 
тому що зараз можна лише припускати, що саме впливає на такий рівень забруднення.
\end{frame}


\begin{frame}
  \frametitle{Чи впливає швидкість вітру (windspeed) на концентрацію частинок PM2.5 і PM10?}

  Ні, на жаль, швидкість вітру має не значний вплив на концентрацію цих частинок, так як це тверді речовини. 

\end{frame}

\begin{frame}
  \frametitle{Як зміни в концентрації  $O_3$  та $SO_2$ впливають на загальний рівень забруднення повітря (AQI)?}

  Концентрація $O_3$ має більший вплив на загальний рівень AQI ніж $SO_2$.
\end{frame}

\begin{frame}
  \frametitle{Як змінюється якість повітря (status) протягом доби в різних районах?}

  Протягом доби якість повітря не зазнає значних змін. В загальному є незначне покращення пообіді.

\end{frame}

\begin{frame}
  \frametitle{Які регіони (county) мають найвищий середній рівень забруднення повітря (AQI) протягом року?}

  Найвищий середній рівень забруднення повітря (AQI) протягом року мають такі регіони: 
  \begin{multicols}{2}
    \begin{itemize}
        \item Kinmen County
        \item Kaohsiung City
        \item Lienchiang County
        \item Tainan City
        \item Chiayi County
        \item Nantou County
        \item Chiayi City
        \item Changhua County
        \item Yunlin County
        \item Taichung City
    \end{itemize}
    \end{multicols} 

\end{frame}

\begin{frame}
  \frametitle{Як змінився загальний рівень забруднення по регіонам після початку реформи?}

  Загальний рівень AQI по регіонам зменшується, тобто показники покращуються після початку реформи.
  Більш явні зміни помітні через декілька років, після початку, що є цілком логічним. 
  Якщо уряд продовжить вводити обмеження та покращувати систему реформ, то показники в усій республіці нормалізуються. 

\end{frame}

\begin{frame}
  \frametitle{Чи існує залежність між початком реформ та показниками забруднення??}

  Після побудови низки гарфіків, було відмічено, що після реформи суттєво змінився 
  лише показник концентрації $S0_2$ та незначні зміними $NO$. Всі інші показники, не зазнали суттєвих змін.
  
\end{frame}

\begin{frame}
  \frametitle{Як змінюється якість повітря залежно від станції виміру у містах?}

  Якість повітря змінюється нерівномірно у містах. Тобто саме від станції виміру не залежить, 
    на це впливають інші фактори, які зазначені вище.

\end{frame}

% Частина про довірчі інтервали, графік таблиці, висновки по кожному з 5 блоків 

\begin{frame}
  \section{Довірчі інтервали}

  \frametitle{Зміст}
  \tableofcontents[currentsection]
\end{frame}

\begin{frame}
  \frametitle{Розподіл довірчих інтервалів}
  Для цiлей цiєї лабораторної роботи всi статистики можна подiлити на три групи:
  \begin{enumerate}
    \renewcommand{\theenumi}{\Roman{enumi}}
    \item  статистики, для яких загальновiдомо, що вони мають асимптотично нормальний розподiл,
    дисперсiю якого можна просто оцiнити. 
    %До таких статистик належать вибiркове середнє, вибiркова дисперсiя,рiзницi середнiх, частка спостережень, якi задовольняють певний критерiй;
     
    \item статистики, для яких загальновiдомо, що вони мають асимптотично нормальний розподiл, але для
    яких асимптотичну дисперсiю важко оцiнити. 
    %До таких статистик належать у першу чергу медiани та iншi квантилi (дисперсiя залежить вiд невiдомої щiльности розподiлу);

    \item  статистики, асимптотичний розподiл яких у загальному випадку не є вiдомий. До таких статистик
    належать коефiцiєнти кореляцiї Пiрсона та Спiрмана та деякi iншi.
  \end{enumerate}
\end{frame} 

\begin{frame}
  \frametitle{Розподіл довірчих інтервалів}
  Було досліджено числові дані з таких колонок:
  \begin{multicols}{2}
    \begin{itemize}
        \item AQI
        \item $SO_2$
        \item $CO$
        \item $O_3$
        \item $pm_{10}$
        \item $pm_{2.5}$
        \item $NO_2$
        \item $NO_x$
        \item $NO$
        \item windspeed
        \item winddirection
    \end{itemize}
    \end{multicols} 
  Було обчислено довірчі інтервали для статистик I, II, III груп.
\end{frame}  

\begin{frame}
  \frametitle{Розподіл довірчих інтервалів}
  Додатково було досліджено такі питання:
  \begin{itemize}
    \item Як змінюється якість повітря (AQI): 
    \begin{itemize}
        \item впродовж доби
        \item по місяцях (сезонам)
        \item по регіонам до та після реформи
        \item по регіонам за 2016-2017 та 2023-2024 роки
    \end{itemize}
  \end{itemize}
  
\end{frame}

\begin{frame}
  \frametitle{Перша група}

  
\end{frame}

\begin{frame}
  \frametitle{Друга група}

  
\end{frame}

\begin{frame}
  \frametitle{Третя група}

  
\end{frame}



  
% Блок про формулювання/тестування гiпотез, якi доречнi за наслiдком проведеного ранiше розвiдкового аналiзу.

\begin{frame}
  \section{Гіпотези}

  \frametitle{Зміст}
  \tableofcontents[currentsection]
\end{frame}

\begin{frame}
  \frametitle{Формування гіпотез}

  
\end{frame}

\begin{frame}
  \frametitle{Тестування гіпотез}

  
\end{frame}

% Висновки по ЛР2

\begin{frame}
  \section{Висновок}

  \frametitle{Зміст}
  \tableofcontents[currentsection]
\end{frame}

\begin{frame}
  \frametitle{Узагальнений висновок}
  Статичтичні методи, які були використані в цій лабораторній роботі, 
  дозволяють грунтовно оцінити якість тих чи інших показників якости повітря у регіонах Тайваню.
  
\end{frame}

\end{document}
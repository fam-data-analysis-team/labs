\documentclass{beamer}

\usepackage[utf8]{inputenc}   % Підтримка UTF-8
\usepackage[ukrainian]{babel} % Підтримка української мови
\usepackage[ukrainian=nohyphenation]{hyphsubst}
\usepackage{booktabs}
\usepackage[T2A]{fontenc}      % Кодова таблиця для кирилиці
\usepackage{amsmath, amsfonts} % Для математики, якщо потрібно
\usepackage{hyperref}          % Для створення посилань
\usepackage{listings}          % Пакет для вставки коду
\usepackage{graphicx}
\usepackage{csvsimple}
\usepackage{parskip}
\usepackage{csquotes}
\usepackage{xcolor}
\usepackage{multicol} % Для багатостовпчикового тексту

\usepackage{tabularray}
\usepackage{float}
\usepackage{codehigh}
\usepackage[normalem]{ulem}

\UseTblrLibrary{booktabs}
\UseTblrLibrary{siunitx}
\newcommand{\tinytableTabularrayUnderline}[1]{\underline{#1}}
\newcommand{\tinytableTabularrayStrikeout}[1]{\sout{#1}}
\NewTableCommand{\tinytableDefineColor}[3]{\definecolor{#1}{#2}{#3}}

\usetheme{Madrid}

% Прибираєм навігацію з кожного слайду
\beamertemplatenavigationsymbolsempty

\title{Лабораторна робота №4}
\subtitle{Непараметрична регресiя. Аналіз головних компонент}
\subtitle{Команда №9}

% [], щоб прибрати імена з кожного слайду
\author[]{
  Баранівська В.О.,
  Корсун Є. В.,
  Хмарук О. Ю.,
  Літковський А.С.,
  Кудін Н. А.
}
\date{2025}

\begin{document}

\frame{\titlepage}

\graphicspath{{../../../}} % Ensure this path is correct or remove it if not needed

%Короткий підсумок ЛР 1-2 (якщо є свіжі погляди, можна ще зменшити/додати/змінити)

\begin{frame}
  \section{Набір даних}

  \frametitle{Зміст}
  \tableofcontents[currentsection]
\end{frame}

\begin{frame}
  \frametitle{Набір даних}

  Було вирішено дослідити якість повітря Тайваню. Уряд провінції намагається
  контролювати та покращувати якість повітря. Тому 17 грудня 2017 року була введена
  реформа \textit{Air Pollution Control Action Plan}.

  \begin{center}

  \end{center}
\end{frame}

\begin{frame}
  \frametitle{Висновки EDA}
  \begin{itemize}
    \item Загальний рівень AQI по регіонам зменшується, тобто показники покращуються після початку реформи.
    Більш явні зміни очікувано помітні через декілька років після реформи.

    \item Якість повітря змінюється нерівномірно у містах.
  \end{itemize}
\end{frame}

\begin{frame}
  \frametitle{Питання для дослідження}
  Як реформа покращення якості повітря вплинула на якість повітря,
   враховуючи можливі нелінійні залежності та взаємодії між факторами?
\end{frame}

\begin{frame}
  \section{Підготовка даних}

  \frametitle{Зміст}
  \tableofcontents[currentsection]
\end{frame}

% Описати, чи використовується той самий агрегований (по днях, медіани) датасет, що і в ЛР3.
% Зазначити, чи потрібна додаткова обробка (наприклад, робота з викидами, якщо вони сильно впливають на непараметричні оцінки).
\begin{frame}
  \frametitle{Підготовка даних для непараметричної регресії}
  \begin{itemize}
    \item \textbf{Використання агрегованих даних:} Для аналізу використовується той самий датасет, що і в ЛР3, агрегований по днях з використанням медіанних значень для зменшення впливу викидів та шуму в погодинних даних.
    \item \textbf{Цільова змінна:} AQI (Air Quality Index).
    \item \textbf{Основні регресори для непараметричного аналізу:}
        \begin{itemize}
            \item \texttt{reform\_days}: кількість днів з моменту початку реформи (для оцінки впливу реформи).
            \item \texttt{windspeed}: швидкість вітру (для дослідження нелінійного впливу погодних умов).
            \item \texttt{july\_days}: кількість днів з початку липня (для врахування сезонності).
        \end{itemize}
    \item \textbf{Обробка викидів:} Оскільки непараметричні методи, особливо ядрові, можуть бути чутливими до екстремальних значень, важливо перевірити наявність впливових викидів. 
    \item \textbf{Масштабування даних:} Для деяких ядрових методів та методів вибору ширини вікна може бути корисним масштабування регресорів (наприклад, стандартизація), особливо якщо вони мають різні порядки величин.
  \end{itemize}
\end{frame}

\begin{frame}
  \section{Ядрова регресія}

  \frametitle{Зміст}
  \tableofcontents[currentsection]
\end{frame}

\begin{frame}
  \frametitle{Ядрова регресія: Основні ідеї}
  %Ядрова регресія — це непараметричний метод оцінки умовної математичної сподіваності залежної змінної $Y$ при заданому значенні незалежної змінної $X=x$.
  \begin{itemize}
    \item \textbf{Оцінка Надарая-Вотсона (Nadaraya-Watson Estimator):}
    Найбільш поширена форма ядрової регресії. Оцінка функції регресії $m(x) = E[Y|X=x]$ в точці $x$ задається як:
    $$ \hat{m}_h(x) = \frac{\sum_{i=1}^{n} K_h(x - X_i) Y_i}{\sum_{i=1}^{n} K_h(x - X_i)} $$
    де:
    \begin{itemize}
        \item $n$ – кількість спостережень.
        \item $Y_i$ – значення залежної змінної для $i$-го спостереження.
        \item $X_i$ – значення незалежної змінної для $i$-го спостереження.
        \item $K_h(u) = \frac{1}{h}K(\frac{u}{h})$ – масштабоване ядро.
        \item $K(\cdot)$ – ядрова функція (kernel function).
        \item $h$ – параметр згладжування (bandwidth або ширина вікна).
    \end{itemize}
  \end{itemize}
\end{frame}

\begin{frame}
  \frametitle{Ядрова регресія: Основні ідеї(2)}
  \begin{itemize}
    \item \textbf{Локальна поліноміальна регресія (Local Polynomial Regression):}
    Узагальнення оцінки Надарая-Вотсона. Замість локального усереднення $Y_i$, вона апроксимує функцію регресії $m(x)$ локальним поліномом степеня $p$.
    Для локально-лінійної регресії ($p=1$), оцінка $\hat{\beta}_0$ в точці $x$ є оцінкою $m(x)$:
    $$ (\hat{\beta}_0, \hat{\beta}_1) = \arg\min_{\beta_0, \beta_1} \sum_{i=1}^{n} \left(Y_i - \beta_0 - \beta_1(X_i - x)\right)^2 K_h(X_i - x) $$
    Тоді $\hat{m}_h(x) = \hat{\beta}_0$.
    Локальні поліноми вищих порядків можуть краще адаптуватися до кривизни функції та зменшувати зміщення на границях.
  \end{itemize}
\end{frame}

\begin{frame}
  \frametitle{Побудова моделі (для всіх регресорів одночасно)}
  \textbf{Завдання:} Оцінити $AQI = m(\text{reform\_days, july\_days, windspeed}) + \epsilon$.

  \begin{itemize}
    \item \textbf{Багатовимірна ядрова регресія:}
    Формула Надарая-Вотсона для $d$ регресорів $X = (X_1, \dots, X_d)$:
    $$ \hat{m}_H(x) = \frac{\sum_{i=1}^{n} K_H(x - X_i) Y_i}{\sum_{i=1}^{n} K_H(x - X_i)} $$
    де $x = (x_1, \dots, x_d)$, $K_H(u) = \frac{1}{\det(H)} K(H^{-1}u)$, $H$ – матриця параметрів згладжування (bandwidth matrix).
    Часто використовують добуток одновимірних ядер: $K(u_1, \dots, u_d) = \prod_{j=1}^d K_j(u_j)$ і діагональну матрицю $H$ з $h_1, \dots, h_d$ на діагоналі.

    %\item \textbf{Проблеми та виклики:}
    %\begin{itemize}
    %    \item \textbf{Curse of Dimensionality:}
    %    Зі збільшенням кількості регресорів ($d$) дані стають все більш розрідженими. Для отримання надійної локальної оцінки потрібно експоненційно більше даних.
    %    Якість оцінки швидко погіршується зі зростанням $d$.
    %\end{itemize}

    \item \textbf{Можливий підхід:} Якщо повна багатовимірна модель виявляється непрактичною, доцільно перейти до аналізу впливу кожного ключового регресора окремо або в парах, фіксуючи інші змінні на середніх значеннях або використовуючи адитивні моделі.
  \end{itemize}
\end{frame}


% Опис спроби застосувати ядрову регресію для всіх ключових регресорів (reform_days, july_days, windspeed) одночасно.
% Якщо виникли проблеми (обчислювальна складність, "прокляття розмірності"), обґрунтувати перехід до аналізу впливу кожного регресора окремо або для невеликої підмножини.
\begin{frame}
  \frametitle{Результати тестування}
\end{frame}

\begin{frame}
  \frametitle{Візуалізація}
\end{frame}

%\begin{frame}
%  \frametitle{Побудова моделі (для окремих регресорів)}
%  \textbf{Обґрунтування:} Через проблеми з багатовимірною ядровою регресією, детальніше досліджується вплив окремих ключових регресорів на AQI.
%  \begin{itemize}
%    \item \textbf{Вибір регресорів:}
%        \begin{itemize}
%            \item \texttt{reform\_days}: для оцінки нелінійного ефекту реформи з часом.
%            \item \texttt{windspeed}: для виявлення нелінійної залежності AQI від швидкості вітру (в ЛР3 спостерігалися ознаки такої залежності).
%        \end{itemize}
%    \item \textbf{Модель:} $AQI = m(X) + \epsilon$, де $X$ – один з обраних регресорів.
%    \item \textbf{Тип ядра:} Гауссове ядро $K(u) = \frac{1}{\sqrt{2\pi}} e^{-\frac{1}{2}u^2}$ або ядро Єпанечникова $K(u) = \frac{3}{4}(1 - u^2) \mathbf{1}_{(|u|\le1)}$. Вибір може базуватися на теоретичних властивостях та практичній зручності. Ядро Єпанечникова часто є оптимальним за MSE, Гауссове – гладке.
%    \item \textbf{Вибір ширини вікна ($h$):} Крос-валідація за методом виключення по одному (LOOCV) для пошуку $h$, що мінімізує прогнозовану суму квадратів залишків (PRESS) або середньоквадратичну помилку (MSE) на тестових даних.
%    $$ \text{LOOCV-MSE}(h) = \frac{1}{n} \sum_{i=1}^{n} (Y_i - \hat{m}_{h,-i}(X_i))^2 $$
%  \end{itemize}
%\end{frame}   

\begin{frame}
  \frametitle{Побудова моделі (для окремих регресорів)}
  \begin{itemize}
    \item \textbf{Порівняння оцінок:}
        \begin{itemize}
            \item \textbf{Оцінка Надарая-Вотсона (NW):} Проста у реалізації, але може мати більше зміщення на границях та в областях з високою кривизною.
            \item \textbf{Локально-лінійна регресія (LL):} Зазвичай має кращі властивості щодо зміщення, особливо на границях діапазону даних. Оцінює $\beta_0$ з локальної МНК задачі:
            $$ \min_{\beta_0, \beta_1} \sum_{i=1}^{n} (Y_i - \beta_0 - \beta_1(X_i - x))^2 K_h(X_i - x) $$
        \end{itemize}
  \end{itemize}
\end{frame}

\begin{frame}
  \frametitle{Частково-лінійна модель: Ідея та структура}
  Частково-лінійна модель поєднує переваги параметричних лінійних моделей та непараметричних моделей.
  Вона припускає, що залежність $Y$ від деяких регресорів $X$ є лінійною, а від інших регресорів $Z$ – нелінійною.

  \textbf{Структура моделі:}
  $$ Y_i = X_i^T \beta + m(Z_i) + \epsilon_i $$
  де:
  \begin{itemize}
    \item $Y_i$ – залежна змінна.
    \item $X_i = (X_{i1}, \dots, X_{ip})^T$ – вектор регресорів, що входять в модель лінійно.
    \item $\beta = (\beta_1, \dots, \beta_p)^T$ – вектор невідомих параметрів для лінійної частини.
    \item $Z_i$ – регресор (або вектор регресорів), що входить в модель нелінійно. Для простоти часто розглядають одновимірний $Z_i$.
    \item $m(\cdot)$ – невідома гладка функція.
    \item $\epsilon_i$ – випадкова похибка, $E[\epsilon_i|X_i, Z_i] = 0$.
  \end{itemize}
\end{frame}

%\begin{frame}
%  \textbf{Переваги:}
%  \begin{itemize}
%    \item Дозволяє уникнути "прокляття розмірності", якщо більшість регресорів мають лінійний вплив.
%    \item Забезпечує легку інтерпретацію коефіцієнтів $\beta$ для лінійної частини.
%    \item Моделює нелінійні ефекти для обраних змінних.
%  \end{itemize}
%\end{frame}

% Обґрунтування вибору регресорів для детального непараметричного аналізу (наприклад, reform_days та windspeed, для яких у ЛР3 спостерігалися нелінійності).
% Вибір типу ядра (наприклад, Гауссове, Єпанечникова) та його обґрунтування.
% Метод вибору оптимальної ширини вікна (bandwidth) – крос-валідація (LOOCV або K-fold). Описати процедуру.
% Порівняння оцінок Надарая-Вотсона та локальної поліноміальної регресії (наприклад, локально-лінійної).
\begin{frame}
  \frametitle{Результати тестування}


\end{frame}

\begin{frame}
  \frametitle{Візуалізація}


\end{frame}


\begin{frame}
  \section{Частково-лінійна модель}

  \frametitle{Зміст}
  \tableofcontents[currentsection]
\end{frame}

\begin{frame}
  \frametitle{Розподіл регресорів}


\end{frame}

\begin{frame}
  \frametitle{Оцінка регресорів}

\end{frame}

\begin{frame}
  \frametitle{Візуалізація}


\end{frame}

\begin{frame}
  \section{Порівняня результатів ЛР3 та ЛР4}

  \frametitle{Зміст}
  \tableofcontents[currentsection]
\end{frame}

\begin{frame}
  \frametitle{Порівняння результатів ЛР3 (параметрична) та ЛР4 (непараметрична)}
  \begin{itemize}
    \item \textbf{Гнучкість моделі:}
        \begin{itemize}
            \item \textbf{ЛР3 (Параметрична):} Накладала жорсткі функціональні форми (лінійність, поліноми низьких степенів, логарифми). Могла пропустити складні нелінійні залежності.
            \item \textbf{ЛР4 (Непараметрична):} Дозволяє даним "говорити самим за себе" щодо форми залежностей. Ядрова регресія та PLM можуть виявити нелінійності, які були б непомітні в параметричних моделях.
        \end{itemize}
    \item \textbf{Інтерпретація ефектів:}
        \begin{itemize}
            \item \textbf{ЛР3:} Коефіцієнти мають пряму інтерпретацію (наприклад, зміна $Y$ при зміні $X$ на одиницю).
            \item \textbf{ЛР4:} Ефекти часто інтерпретуються візуально через графіки $\hat{m}(X)$. Для PLM, лінійна частина $\hat{\beta}$ інтерпретується як у ЛР3, але для $m(Z)$ – візуально.
        \end{itemize}
    \end{itemize}
\end{frame}

\begin{frame}
  \frametitle{Порівняння результатів ЛР3 (параметрична) та ЛР4 (непараметрична)}
  \begin{itemize}
    \begin{itemize}    
    \item \textbf{Висновки щодо впливу реформи (\texttt{reform\_days}):}
        \begin{itemize}
            \item \textbf{ЛР3:} Який був висновок про ефект реформи (наприклад, лінійний, квадратичний)?
            \item \textbf{ЛР4:} Чи підтверджує непараметричний аналіз (ядрова регресія, $m(\texttt{reform\_days})$ в PLM) попередні висновки, чи виявляє більш складну динаміку ефекту?
        \end{itemize}
    \item \textbf{Висновки щодо впливу швидкості вітру (\texttt{windspeed}):}
        \begin{itemize}
            \item \textbf{ЛР3:} Чи був ефект вітру лінійним, чи використовувалися поліноми/інші трансформації?
            \item \textbf{ЛР4:} Яку форму залежності AQI від \texttt{windspeed} показує ядрова регресія? Чи узгоджується це з фізичними очікуваннями?
        \end{itemize}
  \end{itemize}
  
\end{frame}

\begin{frame}
  \section{Висновок}

  \frametitle{Зміст}
  \tableofcontents[currentsection]
\end{frame}

\begin{frame}
  \frametitle{Висновок}
  % Підсумувати основні результати непараметричного аналізу.
  % - Які нелінійні залежності були виявлені та для яких змінних?
  % - Як непараметричні моделі доповнили/змінили розуміння, отримане з параметричних моделей (ЛР3)?
  % - Чи вдалося за допомогою PLM отримати збалансовану модель з інтерпретованими лінійними ефектами та гнучкою нелінійною компонентою?
  % - Відповідь на головне питання дослідження: Як реформа покращення якості повітря вплинула на якість повітря, враховуючи нелінійності?
  % - Обмеження дослідження та можливі напрямки для подальшої роботи.

  \begin{itemize}
    \item Непараметричний аналіз дозволив дослідити гнучкі форми залежностей AQI від ключових факторів, таких як \texttt{reform\_days} та \texttt{windspeed}, без апріорних припущень про їх функціональну форму.
    %\item \textit{(Тут будуть конкретні висновки про форму залежностей, наприклад: "Ядрова регресія для \texttt{reform\_days} показала, що ефект реформи не є монотонним, а має складнішу динаміку..." або "Залежність AQI від \texttt{windspeed} виявилася U-подібною...")}
    \item Частково-лінійна модель дозволила поєднати інтерпретованість лінійних ефектів для контрольних змінних з гнучкою оцінкою нелінійного впливу \texttt{reform\_days} (або \texttt{windspeed}).
    \item Порівняння з результатами ЛР3 покаже, наскільки параметричні моделі були адекватними та чи непараметричні підходи надали суттєво нову інформацію або покращили якість моделювання.
    %\item \textbf{Щодо впливу реформи:} \textit{(Остаточний висновок на основі всіх моделей, наприклад: "Реформа мала статистично значущий вплив на покращення якості повітря, причому цей вплив змінювався нелінійно з часом...")}
  \end{itemize}

\end{frame}


\end{document}
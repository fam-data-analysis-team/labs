\documentclass{beamer}

\usepackage[utf8]{inputenc}   % Підтримка UTF-8
\usepackage[ukrainian]{babel} % Підтримка української мови
\usepackage[ukrainian=nohyphenation]{hyphsubst}
\usepackage{booktabs}
\usepackage[T2A]{fontenc}      % Кодова таблиця для кирилиці
\usepackage{amsmath, amsfonts} % Для математики, якщо потрібно
\usepackage{hyperref}          % Для створення посилань
\usepackage{listings}          % Пакет для вставки коду
\usepackage{graphicx}
\usepackage{csvsimple}
\usepackage{parskip}
\usepackage{csquotes}
\usepackage{xcolor}
\usepackage{multicol} % Для багатостовпчикового тексту

\usetheme{Madrid}

% Прибираєм навігацію з кожного слайду
\beamertemplatenavigationsymbolsempty

\title{Лабораторна робота №3}
\subtitle{Регресійний аналіз}
\subtitle{Команда №9}

% [], щоб прибрати імена з кожного слайду
\author[]{
  Баранівська В.О.,
  Корсун Є. В.,
  Хмарук О. Ю.,
  Літковський А.С.,
  Кудін Н. А.
}
\date{2025}

\begin{document}

\frame{\titlepage}

\graphicspath{{../../../}} % Ensure this path is correct or remove it if not needed

%Короткий підсумок ЛР 1-2 (якщо є свіжі погляди, можна ще зменшити/додати/змінити)

\begin{frame}
  \frametitle{Набір даних}

  Було вирішено дослідити якість повітря Тайваню. Уряд провінції намагається
  контролювати та покращувати якість повітря. Тому 17 грудня 2017 року була введена
  реформа \textit{Air Pollution Control Action Plan}.

  \begin{center}

  \end{center}
\end{frame}

\begin{frame}
  \frametitle{Набір даних}

  Загальний опис датасету:

  \begin{enumerate}
    \item Кількість рядків: 5\,882\,208
    \item Кількість стовпців: 25

    \begin{itemize}
      \item Числові: 19
      \item Факторні: 4
      \item Дата: 1
      \item Інші: 1
    \end{itemize}
  \end{enumerate}
\end{frame}

\begin{frame}
  \section{Висновки EDA та гіпотез}

  \frametitle{Зміст}
  \tableofcontents[currentsection]
\end{frame}

\begin{frame}
  \frametitle{Висновки EDA}
   \begin{itemize}

  \item Загальний рівень AQI по регіонам зменшується, тобто показники покращуються після початку реформи.
  Більш явні зміни помітні через декілька років, після початку, що є цілком логічним.
  \item Якість повітря змінюється нерівномірно у містах.

  \end{itemize}
\end{frame}

\begin{frame}
  \frametitle{Висновки з гіпотез та довірчих інтервалів}
  \begin{itemize}
  \item 
  \item 

  \end{itemize}
\end{frame}

\begin{frame}
  \section{Моделювання}

  \frametitle{Зміст}
  \tableofcontents[currentsection]
\end{frame}

\begin{frame}
  \frametitle{Питання для дослідження}
  Як метеорологічні фактори (наприклад, швидкість вітру) та концентрації окремих 
  забруднювачів впливають на Індекс якості повітря (AQI) 
  враховуючи регіональні відмінності та ефект реформи року?

  \begin{itemize}
    \item Залежна змінна: AQI
    \item Незалежні змінні: windspeed, pm2.5, pm10, after-reform 
    \item Контрольні змінні:  
  \end{itemize}
\end{frame}

\begin{frame}
  \frametitle{Пропущені дані по змінним}
   
\end{frame}

\begin{frame}
  \frametitle{Моделювання(1)}
   $$AQI \sim windspeed $$
   $\beta_{windspeed}$: Очікується негативний знак. 
  Сильніший вітер зазвичай сприяє кращому розсіюванню забруднювачів, 
  що призводить до зниження AQI (покращення якості повітря).
\end{frame}

\begin{frame}
  \frametitle{Моделювання(2)}
   $$AQI \sim windspeed + after\_reform $$
   $\beta_{after\_reform}$:Очікується негативний знак. 
   Метою реформи, ймовірно, було покращення якості повітря,
   тому після її впровадження AQI мав би знизитися.
\end{frame}

\begin{frame}
  \frametitle{Моделювання(3)}
   $$AQI \sim windspeed + after\_reform + ... $$
   $\beta_{...}$:Очікується позитивний знак. 
   знак для кожного забруднювача. Вищі концентрації цих речовин 
   безпосередньо погіршують якість повітря та підвищують AQI.
\end{frame}

\begin{frame}
  \subsection{Перевірка на стійкість}

  \frametitle{Зміст}
  \tableofcontents[currentsection]
\end{frame}

\begin{frame}
  \frametitle{Перевірка на стійкість(1)}
   
\end{frame}

\begin{frame}
  \subsection{Оцінка якості моделі}

  \frametitle{Зміст}
  \tableofcontents[currentsection]
\end{frame}

\begin{frame}
  \frametitle{Оцінка якості моделі(1)}
   
\end{frame}

\begin{frame}
  \section{Кінцева модель та її оцінка}

  \frametitle{Зміст}
  \tableofcontents[currentsection]
\end{frame}

\begin{frame}
  \frametitle{Кінцева модель та її оцінка}
   
\end{frame}

\begin{frame}
  \section{Висновок}

  \frametitle{Зміст}
  \tableofcontents[currentsection]
\end{frame}

\begin{frame}
  \frametitle{Висновок}
   
\end{frame}
\end{document}
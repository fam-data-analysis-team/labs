\documentclass{beamer}

\usepackage[utf8]{inputenc}   % Підтримка UTF-8
\usepackage[ukrainian]{babel} % Підтримка української мови
\usepackage[ukrainian=nohyphenation]{hyphsubst}
\usepackage{booktabs}
\usepackage[T2A]{fontenc}      % Кодова таблиця для кирилиці
\usepackage{amsmath, amsfonts} % Для математики, якщо потрібно
\usepackage{hyperref}          % Для створення посилань
\usepackage{listings}          % Пакет для вставки коду
\usepackage{graphicx}
\usepackage{csvsimple}
\usepackage{parskip}
\usepackage{csquotes}
\usepackage{xcolor}
\usepackage{multicol} % Для багатостовпчикового тексту

\usetheme{Madrid}

% Прибираєм навігацію з кожного слайду
\beamertemplatenavigationsymbolsempty

\title{Лабораторна робота №3}
\subtitle{Регресійний аналіз}
\subtitle{Команда №9}

% [], щоб прибрати імена з кожного слайду
\author[]{
  Баранівська В.О.,
  Корсун Є. В.,
  Хмарук О. Ю.,
  Літковський А.С.,
  Кудін Н. А.
}
\date{2025}

\begin{document}

\frame{\titlepage}

\graphicspath{{../../../}} % Ensure this path is correct or remove it if not needed

%Короткий підсумок ЛР 1-2 (якщо є свіжі погляди, можна ще зменшити/додати/змінити)

\begin{frame}
  \section{Набір даних}

  \frametitle{Зміст}
  \tableofcontents[currentsection]
\end{frame}

\begin{frame}
  \frametitle{Набір даних}

  Було вирішено дослідити якість повітря Тайваню. Уряд провінції намагається
  контролювати та покращувати якість повітря. Тому 17 грудня 2017 року була введена
  реформа \textit{Air Pollution Control Action Plan}.

  \begin{center}

  \end{center}
\end{frame}

\begin{frame}
  \frametitle{Набір даних}

  Загальний опис датасету:

  \begin{enumerate}
    \item Кількість рядків: 5\,882\,208
    \item Кількість стовпців: 25

    \begin{itemize}
      \item Числові: 19
      \item Факторні: 4
      \item Дата: 1
      \item Інші: 1
    \end{itemize}
  \end{enumerate}
\end{frame}

\begin{frame}
  \frametitle{Висновки EDA}
   \begin{itemize}

  \item Загальний рівень AQI по регіонам зменшується, тобто показники покращуються після початку реформи.
  Більш явні зміни помітні через декілька років, після початку, що є цілком логічним.
  \item Якість повітря змінюється нерівномірно у містах.

  \end{itemize}
\end{frame}

% \begin{frame}
%   \frametitle{Висновки з гіпотез та довірчих інтервалів}
%   \begin{itemize}
%   \item 
%   \item 
% 
%   \end{itemize}
% \end{frame}

\begin{frame}
  \section{Моделювання}

  \frametitle{Зміст}
  \tableofcontents[currentsection]
\end{frame}

\begin{frame}
  \frametitle{Питання для дослідження}
  Як реформа покращення якості повітря вплинула на якість повітря?

  
\end{frame}

\begin{frame}
  \frametitle{Питання для дослідження}

  Даний набір даних є панельним. Для аналізу будемо використовувати модель з фіксованими ефектами.

  \begin{itemize}
    \item Залежна змінна: AQI
    \item Незалежні змінні: after\_reform\footnotemark, windspeed
    \item Контрольні змінні:  
    \item Фіксовані ефекти: sitename
  \end{itemize}

  \footnotetext[1]{Довірнює 1, якщо дата після 17 грудня 2017 року, 0 в іншому випадку}
\end{frame}

% \begin{frame}
%   \frametitle{Пропущені дані по змінним}
%    
% \end{frame}

\begin{frame}
  \frametitle{Моделювання}

   $$AQI \sim \beta_1 \, \text{after\_reform} + \beta_2 \, \text{windspeed} + \alpha $$

  \begin{enumerate}
    \item $\beta_1$: Очікується негативний знак. 
    Метою реформи було покращення якості повітря,
    тому після її впровадження AQI мав би знизитися.

    \item $\beta_2$: Очікується негативний знак. 
    Сильніший вітер зазвичай сприяє кращому розсіюванню забруднювачів, 
    що призводить до зниження AQI (покращення якості повітря).
  \end{enumerate}
\end{frame}

\begin{frame}
  \subsection{Перевірка на стійкість}

  \frametitle{Зміст}
  \tableofcontents[currentsection]
\end{frame}

\begin{frame}
  \frametitle{Перевірка на стійкість(1)}
   
\end{frame}

\begin{frame}
  \subsection{Оцінка якості моделі}

  \frametitle{Зміст}
  \tableofcontents[currentsection]
\end{frame}

\begin{frame}
  \frametitle{Оцінка якості моделі(1)}
   
\end{frame}

\begin{frame}
  \section{Кінцева модель та її оцінка}

  \frametitle{Зміст}
  \tableofcontents[currentsection]
\end{frame}

\begin{frame}
  \frametitle{Кінцева модель та її оцінка}
   
\end{frame}

\begin{frame}
  \section{Висновок}

  \frametitle{Зміст}
  \tableofcontents[currentsection]
\end{frame}

\begin{frame}
  \frametitle{Висновок}
   
\end{frame}
\end{document}
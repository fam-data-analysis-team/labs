%! Author = help pls
%! Date = 21.05.2025

\documentclass{article}
\usepackage[utf8]{inputenc} % Підтримка UTF-8
\usepackage[ukrainian]{babel} % Підтримка української мови
\usepackage[ukrainian=nohyphenation]{hyphsubst} % Без переносу слів в українській
\setlength{\emergencystretch}{3cm}
\usepackage{booktabs}
\usepackage[T2A]{fontenc} % Кодова таблиця для кирилиці
\usepackage{amsmath, amsfonts} % Для математики, якщо потрібно
\usepackage[a4paper, left=3cm, right=1.5cm, top=2cm, bottom=2cm]{geometry}
\usepackage{fancyhdr}        % Пакет для налаштування колонтитулів
\usepackage{hyperref}        % Для створення посилань
\usepackage{listings}          % Пакет для вставки коду
\usepackage{graphicx}
\usepackage{subcaption}  % Пакет для відображення кількох картинок в одному рядку
\usepackage{csvsimple}
\usepackage{parskip}
\usepackage{csquotes}
\usepackage{pgfplotstable}
\usepackage{xcolor}      % For custom colors
\usepackage{colortbl}

\pagestyle{fancy}            % Встановлюємо fancy стиль для колонтитулів
\fancyhf{}                   % Очищуємо всі колонтитули
% Встановлюємо посилання на зміст у лівий верхній кут
\fancyhead[L]{\hyperlink{mytarget}{Зміст}}  % Посилання на зміст
% Встановлюємо нумерацію сторінок у правий верхній кут
\fancyhead[R]{\thepage}
% Прибираємо колонтитули з першої сторінки
\thispagestyle{plain}
% Прибираємо колонтитули для першої сторінки
\fancypagestyle{plain}{
    \fancyhf{}  % Очищуємо колонтитули
    \renewcommand{\headrulewidth}{0pt}  % Вимикаємо лінію колонтитулів
}
\hypersetup{
    colorlinks=true,        % Enable colored links
    linkcolor=red!50!black,      % Color for internal links
    citecolor=red!50!black,      % Color for citations
    filecolor=red!50!black,      % Color for file links
    urlcolor=red!50!black        % Color for external URLs
}

\graphicspath{{../../}}


\newcommand\warningsign{⚠}


\begin{document}
% Титульна сторінка
\newpage 
\begin{center}
    \LargeНАЦІОНАЛЬНИЙ ТЕХНІЧНИЙ УНІВЕРСИТЕТ УКРАЇНИ\\
    \Large«КИЇВСЬКИЙ ПОЛІТЕХНІЧНИЙ ІНСТИТУТ\\
    \LargeІМЕНІ ІГОРЯ СІКОРСЬКОГО»\\
    \vspace{1cm}
    Факультет прикладної математики\\
    Кафедра прикладної математики\\
    \vspace{3cm}
    \textbf{Звіт}\\
    \vspace{0.5cm}
    із лабораторної роботи №3\\
    із дисципліни «Аналіз даних»\\
    \vspace{1cm}
    на тему\\
    \textit{Регресійний аналіз}\\
    \vspace{1.5cm}
    \textbf{Команда № 9}\\
    \vspace{2cm}
    \begin{tabbing}
        Виконали: \hspace{10cm} \= Керівник:\\
        студенти групи КМ-23: \> \textit{Тавров Д.Ю.}\\
        \textit{Баранівська В.О.} \> \textit{Доцент,}\\
        \textit{Корсун Є. В.} \> \textit{канд. тех.-наук}.\\
        \textit{Хмарук О. Ю.} \> \\
        \textit{Літковський А.С.} \>\\
        \textit{Кудін Н. А.} \> \\
    \end{tabbing}
    \vspace{3cm}
    \largeКиїв — 2025\\
\end{center}

% Створюємо зміст
\newpage
\hypertarget{mytarget}{} % Якір на сторінці 5
\tableofcontents
\newpage
\section{Особливості датасету}
\subsection{Опис}

\begin{enumerate}
    \item Кількість рядків: 5\,882\,208
    \item Кількість стовпців: 25

    \begin{itemize}
      \item Числові: 19
      \item Факторні: 4
      \item Дата: 1
      \item Інші: 1
    \end{itemize}
  \end{enumerate}

\subsection{Візуалізація}
\subsection{Висновки EDA та гіпотез}

Отже, після розвідкового аналізу та побудови графіків, можна дати відповіді на наступні питання:
\begin{enumerate}
    \item Чи впливає швидкість вітру (windspeed) на концентрацію частинок PM2.5 і PM10?
    
    Ні, на жаль, швидкість вітру має не значний вплив на концентрацію цих частинок, так як це тверді речовини. 
    
    \item Як зміни в концентрації  $O_3$  та $SO_2$ впливають на загальний рівень забруднення повітря (AQI)?
    
    Концентрація $O_3$ має більший вплив на загальний рівень AQI ніж $SO_2$.

    \item Як змінюється якість повітря (status) протягом доби в різних районах?
    
    Протягом доби якість повітря не зазнає значних змін. В загальному є незначне покращення пообіді. 

    \item Які регіони (county) мають найвищий середній рівень забруднення повітря (AQI) протягом року?
     
    Найвищий середній рівень забруднення повітря (AQI) протягом року мають такі регіони: 
    \begin{itemize}
        \item Kinmen County	 
        \item Lienchiang County
        \item Chiayi County	 
        \item Chiayi City	     
        \item Yunlin County	 
        \item Kaohsiung City	 
        \item Tainan City	     
        \item Nantou County	 
        \item Changhua County	 
        \item Taichung City	 
    \end{itemize} 

    \item Як змінився загальний рівень забруднення по регіонам після початку реформи?
    
    Загальний рівень AQI по регіонам зменшується, тобто показники покращуються після початку реформи.
    Більш явні зміни помітні через декілька років, після початку, що є цілком логічним. 
    Якщо уряд продовжить вводити обмеження та покращувати систему реформ, то показники в усій республіці нормалізуються. 

    \item Чи існує залежність між початком реформ та показниками забруднення?
    
    Після побудови низки гарфіків, було відмічено, що після реформи суттєво змінився 
    лише показник концентрації $S0_2$ та незначні зміними $NO$. Всі інші показники, не зазнали суттєвих змін.
    
    \item Як змінюється якість повітря залежно від станції виміру у містах?
    
    Якість повітря змінюється нерівномірно у містах. Тобто саме від станції виміру не залежить, 
    на це впливають інші фактори, які зазначені вище.
\end{enumerate}

\textbf{Гіпотези}
\begin{enumerate}
    \item   Під час виконання лабораторної роботи №1 було помічено,що у другій половині дня AQI трохи вище, ніж у першій.
    Отримали низьке $p$-value і довірчий інтервал, який не включає нуль, 
    тому відкидаємо гіпотезу $H_0$. Це приклад того, коли різниця медіан є 
    статистично значущою, проте її значення, насправді дуже мале.

    \item Також ми спостерігали зниження рівня AQI у період з червня до вересня.
    Отримали дуже низьке $p-value$ і довірчий інтервал, 
    який не включає нуль, тому відкидаємо гіпотезу $H_0$. 
    Різниця медіан є статистично значущою, і її значення, 
    на відміну від попереднього випадку, суттєве.

    \item Перевіримо гіпотезу, що введення реформи позитивно вплинуло на всі види забруднювачів.
     Для всіх забруднювачів $p-value$ близьке до нуля, тому можна з впевненістю сказати, що всі концентрації зменшились після реформи.
  

\end{enumerate}

\newpage
\section{Моделювання}
Досліджуване питання: 

Як метеорологічні фактори (наприклад, швидкість вітру) та концентрації окремих 
забруднювачів впливають на Індекс якості повітря (AQI) 
враховуючи регіональні відмінності та ефект реформи року?


\begin{enumerate}
    \item Вибір залежної та незалежних змінних
    \begin{itemize}
        \item Залежна змінна (Y): aqi (Індекс якості повітря). Це основний показник, який ми прагнемо пояснити та спрогнозувати на основі інших факторів.
        \item Незалежні змінні (X):
        \begin{itemize}
            \item Метеорологічні фактори:
            \begin{itemize}
                \item windspeed (швидкість вітру): Очікується, що швидкість вітру впливає на розсіювання забруднювачів.
                \item winddirec (напрямок вітру): Напрямок вітру може визначати, чи переносяться забруднювачі від джерел до моніторингових станцій.
            \end{itemize}
            \item Концентрації окремих забруднювачів (so2, co, o3, pm2.5, pm10, no2, nox, no):
            Ці змінні безпосередньо вимірюють рівні конкретних забруднюючих речовин, які, як відомо, впливають на загальну якість повітря та AQI.
        \end{itemize}
        \item after\_reform (бінарна змінна: TRUE/FALSE): Ця змінна дозволить оцінити, чи мала реформа вплив на AQI.
        \item Контроль регіональних відмінностей: sitename (назва станції моніторингу) або county (округ): Ці змінні будуть використовуватися для врахування фіксованих ефектів на рівні станції або округу. Це дозволить контролювати неспостережувані, постійні в часі характеристики кожної місцевості, які можуть впливати на AQI (наприклад, топографія, близькість до специфічних джерел забруднення).
        \item Контроль часових ефектів: date: Ця змінна може бути використана для створення фіктивних змінних року, місяця або сезону для контролю загальних часових трендів або сезонності, що впливають на якість повітря в усіх регіонах.
    \end{itemize}

    \item Аналіз потреби застосування логаритмів
    \begin{itemize}
        \item aqi: Індекс якості повітря може мати несиметричний розподіл. Якщо розподіл сильно скошений, логарифмування може допомогти нормалізувати його та стабілізувати дисперсію похибок. Однак, AQI часто використовується без перетворень.
        \item Концентрації забруднювачів (so2, co, o3, pm2.5, pm10, no2, nox, no): Розподіл концентрацій забруднюючих речовин часто є сильно скошеним вправо (багато низьких значень і декілька високих екстремальних). Логарифмування цих змінних є стандартною практикою, оскільки це:
        \begin{itemize}
            \item Зменшує вплив екстремальних значень.
            \item Дозволяє інтерпретувати коефіцієнти як еластичності або напівеластичності (наприклад, зміна AQI у відсотках при зміні концентрації на один відсоток, якщо і AQI логарифмовано).
            \item Може лінеаризувати зв'язок між концентрацією та AQI.
        \end{itemize}
        \item windspeed: Швидкість вітру також може мати скошений розподіл. Логарифмування може бути доцільним, якщо очікується, що відсоткова зміна швидкості вітру має стабільний вплив на AQI, а не абсолютна зміна.
        \item winddirec: Логарифмування напрямку вітру (0-360 градусів) недоцільне. Цю змінну слід обробляти інакше: наприклад, перетворити на синусні та косинусні компоненти або створити категорійні змінні (наприклад, північний, східний, південний, західний вітер).
        \item Висновок: Перед застосуванням логаритмів слід візуально (гістограми, Q-Q графіки) та статистично (тести на нормальність, коефіцієнт асиметрії) перевірити розподіл кожної неперервної змінної. Логарифмування рекомендовано для змінних із сильним скошенням та для тих, де теоретично очікується мультиплікативний або відсотковий зв'язок із залежною змінною.
    \end{itemize}

    \item Потенційна наявність зміщення від неврахованих змінних (OVB)
    \begin{itemize}
        \item Зміщення від неврахованих змінних (OVB) виникає, якщо пропущена змінна одночасно (1) є детермінантою залежної змінної Y (AQI) і (2) корелює з однією або кількома включеними незалежними змінними $\mathbf{X}$.
        \item Невраховані змінні, що можуть вести до OVB:
        \begin{itemize}
            \item Інші метеорологічні фактори:
            \begin{itemize}
                \item Температура повітря: Впливає на хімічні реакції утворення деяких забруднювачів (наприклад, озону) та на інтенсивність опалення/кондиціонування, що впливає на викиди. Температура може корелювати зі швидкістю вітру та сезонними концентраціями забруднювачів.
                \item Відносна вологість: Впливає на утворення вторинних аерозолів (наприклад, частинок PM2.5). Може корелювати з іншими метеофакторами.
                \item Опади: Можуть "вимивати" забруднювачі з атмосфери. Корелюють із сезонністю та іншими метеофакторами.
                \item Висота граничного шару атмосфери: Визначає об'єм, в якому розсіюються забруднювачі.
            \end{itemize}
            \item Джерела викидів та людська активність:
            \begin{itemize}
                \item Інтенсивність дорожнього руху: Основне джерело NOX, CO, PM2.5. Може корелювати з sitename (міські райони), date (будні/вихідні, сезон відпусток) та after\_reform (якщо реформа вплинула на транспорт).
                \item Промислова активність: Джерело SO2, PM. Може корелювати з sitename та after\_reform.
                \item Спалювання біомаси (наприклад, сільськогосподарські залишки, лісові пожежі): Джерело PM, CO. Має сезонний характер і може корелювати з date, winddirec.
                \item Споживання енергії для опалення/охолодження: Впливає на викиди від ТЕЦ та домогосподарств. Корелює з date (сезонність) та температурою (неврахована).
            \end{itemize}
            \item Топографічні особливості: Рельєф місцевості (гори, долини) впливає на розсіювання забруднювачів. Ці особливості фіксовані для кожного sitename, тому їхній вплив частково контролюватиметься фіксованими ефектами місця.
            \item Транскордонне перенесення забруднювачів: Забруднення може переноситися з інших регіонів або країн. Це може корелювати з winddirec, windspeed та date.
            \item Деталі самої реформи: Змінна after\_reform є бінарною. Якщо реформа впроваджувалася поступово, або її інтенсивність/тип заходів відрізнялися по регіонах чи в часі, то проста бінарна змінна може не повністю відображати її вплив, що призведе до OVB.
            \item Наприклад, якщо реформа (after\_reform) супроводжувалася економічним спадом (неврахована змінна), який сам по собі зменшив промислові викиди та інтенсивність руху (що впливають на pm2.5, no2 і, як наслідок, на aqi), то оцінений коефіцієнт для after\_reform може бути зміщеним, відображаючи не лише прямий ефект політики, а й ефект економічного спаду. Аналогічно, якщо тепліша погода (неврахована) призводить до вищих концентрацій озону (o3) і одночасно пов'язана з певними напрямками вітру (winddirec), то коефіцієнт для winddirec може бути зміщеним.
        \end{itemize}
    \end{itemize}

    \item Контрольні змінні
    \begin{itemize}
        \item Ідеально мати в наявності:
        \begin{itemize}
            \item Дані про викиди від основних джерел (транспорт, промисловість, енергетика, сільське господарство) з високою часовою та просторовою роздільною здатністю.
            \item Повніші метеорологічні дані: температура, вологість, опади, сонячна радіація, висота граничного шару.
            \item Дані про землекористування та топографію для кожної станції.
            \item Дані про щільність населення.
            \item Деталізовані дані про компоненти та етапи впровадження реформи.
        \end{itemize}
        \item Наявні по факту (з датасету):
        \begin{itemize}
            \item Метеорологічні: windspeed, winddirec.
            \item Концентрації забруднювачів: so2, co, o3, pm2.5, pm10, no2, nox, no. (Залежно від специфікації моделі, одні забруднювачі можуть виступати як контрольні при аналізі впливу інших або метеофакторів).
            \item Часові контролі: Змінна date дозволяє створити фіктивні змінні для року, місяця, дня тижня, сезону для контролю часових тенденцій та сезонності.
            \item Регіональні контролі: sitename або county для включення як фіксованих ефектів. Це дозволить контролювати всі стабільні в часі неспостережувані відмінності між локаціями.
        \end{itemize}
    \end{itemize}

    \item Гіпотези щодо значень (знаків) коефіцієнтів структурної моделі
    \begin{itemize}
        \item Припускаючи, що вищий AQI означає гіршу якість повітря, та що модель правильно специфікована і задовольняє умову $\mathbb{E}[e \mid \mathbf{X}] = 0$:
        \begin{itemize}
            \item $\beta_{\text{windspeed}}$ (швидкість вітру): Очікується негативний знак. Сильніший вітер зазвичай сприяє кращому розсіюванню забруднювачів, що призводить до зниження AQI (покращення якості повітря).
            \item $\beta_{\text{pollutant}}$ (для so2, co, o3, pm2.5, pm10, no2, nox, no): Очікується позитивний знак для кожного забруднювача. Вищі концентрації цих речовин безпосередньо погіршують якість повітря та підвищують AQI.
            \item $\beta_{\text{after\_reform}}$ (ефект реформи): Очікується негативний знак. Метою реформи, ймовірно, було покращення якості повітря, тому після її впровадження AQI мав би знизитися.
            % \item $\beta_{\text{winddirec}}$ (напрямок вітру): Знак коефіцієнта(ів) для напрямку вітру важко передбачити однозначно без знання розташування основних джерел забруднення відносно моніторингових станцій. Якщо певний напрямок вітру приносить повітря від великого промислового центру до станції, то коефіцієнт для цього напрямку (або відповідної категорії) буде позитивним. Цю змінну, можливо, доведеться взаємодіяти з sitename або перетворити на кілька бінарних змінних.
        \end{itemize}
        \item Ці гіпотези базуються на загальних знаннях про атмосферні процеси та вплив забруднювачів. Реальні значення та знаки коефіцієнтів можуть відрізнятися залежно від специфіки регіону, типу реформи та взаємодії різних факторів.
    \end{itemize}
\end{enumerate}

\section{Перевірка моделі на стійкість}
Для перевірки моделі на стійкість, крім базової моделі та моделі з додатковими контрольними змінними, слід проаналізувати такі модифікації, фокусуючись на стабільності коефіцієнтів для ключових регресорів (windspeed, основні забруднювачі, after\_reform):

%\subsection{Додавання поліномів вищих порядків}
%\begin{itemize}
%    \item \textbf{Мета:} Врахувати нелінійні зв'язки між неперервними регресорами та AQI.
%    \item \textbf{Застосування до вашої моделі:}
%    \begin{itemize}
%        \item windspeed: Вплив швидкості вітру на AQI може бути нелінійним. Наприклад, дуже слабкий вітер може не розсіювати забруднювачі, а дуже сильний – приносити їх з інших територій або піднімати пил.
%        \item \textbf{Дії:} Візуалізуйте зв'язок AQI та windspeed за допомогою діаграми розсіювання з накладеною згладжуючою кривою (наприклад, LOESS). Якщо спостерігається U-подібна, J-подібна або інша чітка нелінійна залежність, додайте до моделі windspeed$^2$ (квадрат швидкості вітру) та, можливо, windspeed$^3$.
%        \item Концентрації забруднювачів (наприклад, pm2.5, so2): Зв'язок між концентрацією конкретного забруднювача та загальним AQI може бути нелінійним. Наприклад, приріст AQI від кожної додаткової одиниці pm2.5 може бути різним при низьких та високих рівнях концентрації.
%        \item \textbf{Дії:} Аналогічно, побудуйте діаграми розсіювання AQI відносно ключових забруднювачів. Якщо є ознаки нелінійності, розгляньте додавання квадратичних (наприклад, pm2.5$^2$) або кубічних членів.
%    \end{itemize}
%    \item \textbf{Аргументація:} Необхідність додавання поліномів має бути обґрунтована візуальним аналізом діаграм розсіювання, які показують, що простий лінійний зв'язок недостатньо добре описує дані.
%\end{itemize}

%\subsection{Додавання факторів взаємодії між регресорами}
%
%    \begin{itemize}
%        \item windspeed та after\_reform (windspeed $\times$ after\_reform): Ефективність розсіювання забруднювачів вітром могла змінитися після реформи, якщо реформа вплинула на характер або локалізацію джерел викидів.
%        \item Концентрація забруднювача та after\_reform (наприклад, pm2.5 $\times$ after\_reform): Якщо реформа була спрямована на конкретні типи забруднювачів, її вплив на AQI може бути більш вираженим при зміні рівнів саме цих забруднювачів.
%        \item windspeed та концентрація забруднювача (наприклад, windspeed $\times$ pm2.5): Вплив високої концентрації PM2.5 на AQI може бути менш критичним при високій швидкості вітру (краще розсіювання).
%        \item windspeed (або winddirec) та sitename (або характеристики сайту): Ефект вітру може залежати від локальної топографії або міської забудови, які частково враховуються фіксованими ефектами sitename. Якщо є можливість класифікувати сайти (наприклад, "міський", "сільський", "промисловий"), можна створити взаємодію вітру з цими категоріями.
%        \item after\_reform та sitename (або county): Вплив реформи міг відрізнятися між регіонами через відмінності у промисловості, дотриманні норм, вихідних рівнях забруднення тощо.
%        \item winddirec та індикатор джерела забруднення (якщо доступний для станцій): Вплив напрямку вітру стає значущим, якщо відомо, де знаходяться основні джерела забруднення відносно станції моніторингу.
%    \end{itemize}
%    \item \textbf{Аргументація:} Вибір факторів взаємодії має ґрунтуватися на теоретичних припущеннях про те, як різні фактори можуть спільно впливати на якість повітря.
%

\subsection{Взяття логаритмів окремих змінних та ступенів таких логаритмів}
Стабілізувати дисперсію, наблизити розподіл змінної до нормального, моделювати відсоткові зміни або еластичності.
\begin{itemize}
    \item Логарифмування aqi: Якщо розподіл AQI сильно скошений, або якщо очікується, що незалежні змінні мають мультиплікативний вплив на AQI.
    \item Логарифмування концентрацій забруднювачів (наприклад, $\log(\text{so2})$, $\log(\text{pm2.5})$): Це стандартна практика, оскільки концентрації часто мають сильно скошений розподіл і їх вплив може бути пропорційним.
    \item Логарифмування windspeed ($\log(\text{windspeed})$): Якщо очікується, що відсоткова зміна швидкості вітру, а не абсолютна, має стабільний ефект на AQI.
    \item Ступені логарифмів (наприклад, $(\log(\text{pm2.5}))^2$): Якщо після логарифмування зв'язок все ще виглядає нелінійним на діаграмі розсіювання (наприклад, AQI від $\log(\text{pm2.5})$), можна додати квадратичний член від логарифмованої змінної.
\end{itemize}
\textbf{Аргументація:} Потреба в логарифмуванні визначається аналізом розподілу змінних та теоретичними міркуваннями про характер зв'язків. Візуалізація (наприклад, графік AQI від $\log(X)$) може підказати необхідність додавання ступенів логарифмів.


\subsection{Створення нових індикаторних (бінарних) змінних}
\textbf{Мета:} Моделювати порогові ефекти або нелінійності шляхом перетворення неперервних змінних на категорійні.

\begin{itemize}
    \item windspeed: Можна розділити на категорії: "штиль/слабкий вітер", "помірний вітер", "сильний вітер" на основі квантилей розподілу або метеорологічно значущих порогів. Наприклад, створити бінарну змінну is\_high\_wind = (windspeed $>$ поріг).
    \item Концентрації забруднювачів (наприклад, pm2.5): Можна створити бінарні змінні, що відповідають певним рівням небезпеки за стандартами якості повітря. Наприклад, is\_pm2.5\_high = (pm2.5 $>$ рекомендований\_рівень).
    \item winddirec: Якщо ще не зроблено, перетворити напрямок вітру (0-360 градусів) на категорійні змінні (наприклад, "Пн", "ПнСх", "Сх", ..., "ПнЗх"), а потім на набір бінарних змінних, виключивши одну як базову категорію.
\end{itemize}
\textbf{Аргументація:} Створення індикаторних змінних доцільне, якщо є підстави вважати, що ефект змінної різко змінюється при переході через певний поріг, а не поступово.


\subsection{Інтерпретація стійкості та контрольні змінні}
Модель вважатиметься стійкою, якщо коефіцієнти біля ключових регресорів (windspeed, основні забруднювачі, after\_reform) та їх статистична значущість суттєво не змінюються при вищезазначених модифікаціях специфікації. Сильні зміни можуть вказувати на пропущені важливі нелінійності, взаємодії або на те, що початкова модель страждала від OVB, і додані члени (наприклад, поліноми чи взаємодії) частково його врахували.

Категорійні контрольні змінні (наприклад, якщо б winddirec було закодовано як 1=Північ, 2=Схід і т.д.) повинні бути перетворені на набір бінарних (фіктивних) змінних, де кожна представляє одну категорію, за винятком однієї базової категорії для уникнення повної мультиколінеарності.

\newpage
\section{Модель 1}
$$AQI \sim windspeed $$
   $\beta_{windspeed}$: Очікується негативний знак. 
  Сильніший вітер зазвичай сприяє кращому розсіюванню забруднювачів, 
  що призводить до зниження AQI (покращення якості повітря).
\subsection{Перевірка на стійкість}
\subsection{Оцінка якості моделі}

\newpage
\section{Модель 2}
$$AQI \sim windspeed + after\_reform $$
   $\beta_{after\_reform}$:Очікується негативний знак. 
   Метою реформи, ймовірно, було покращення якості повітря,
   тому після її впровадження AQI мав би знизитися.
\subsection{Перевірка на стійкість}
\subsection{Оцінка якості моделі}

\newpage
\section{Модель 2}
$$AQI \sim windspeed + after\_reform + якісь забруднювачі$$
   $\beta_{after\_reform}$:Очікується негативний знак. 
   Метою реформи, ймовірно, було покращення якості повітря,
   тому після її впровадження AQI мав би знизитися.
\subsection{Перевірка на стійкість}
\subsection{Оцінка якості моделі}

\newpage
\section{Фінальна модель}
$$AQI \sim windspeed + after\_reform $$
\subsubsection{Перевірка на стійкість}
\subsubsection{Оцінка якості моделі}

\newpage
\section{Висновок}

\newpage
\section{Додаток}
Покликання на репозиторій з кодом:
\url{}{}

\end{document}
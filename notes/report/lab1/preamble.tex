\usepackage[utf8]{inputenc} % Підтримка UTF-8
\usepackage[ukrainian]{babel} % Підтримка української мови
\usepackage[ukrainian=nohyphenation]{hyphsubst} 
\usepackage{booktabs}
\usepackage[T2A]{fontenc} % Кодова таблиця для кирилиці
\usepackage{amsmath, amsfonts} % Для математики, якщо потрібно
\usepackage[a4paper, left=3cm, right=1.5cm, top=2cm, bottom=2cm]{geometry}
\usepackage{fancyhdr}        % Пакет для налаштування колонтитулів
\usepackage{hyperref}        % Для створення посилань
\usepackage{listings}          % Пакет для вставки коду
\usepackage{graphicx}
\usepackage{csvsimple}
\usepackage{parskip}
\usepackage{csquotes}
\usepackage{pgfplotstable}
\usepackage{xcolor}      % For custom colors

\pagestyle{fancy}            % Встановлюємо fancy стиль для колонтитулів
\fancyhf{}                   % Очищуємо всі колонтитули
% Встановлюємо посилання на зміст у лівий верхній кут
\fancyhead[L]{\hyperlink{mytarget}{Зміст}}  % Посилання на зміст
% Встановлюємо нумерацію сторінок у правий верхній кут
\fancyhead[R]{\thepage}
% Прибираємо колонтитули з першої сторінки
\thispagestyle{plain}
% Прибираємо колонтитули для першої сторінки
\fancypagestyle{plain}{
    \fancyhf{}  % Очищуємо колонтитули
    \renewcommand{\headrulewidth}{0pt}  % Вимикаємо лінію колонтитулів
}
\hypersetup{
    colorlinks=true,        % Enable colored links
    linkcolor=red!50!black,      % Color for internal links
    citecolor=red!50!black,      % Color for citations
    filecolor=red!50!black,      % Color for file links
    urlcolor=red!50!black        % Color for external URLs
}

\graphicspath{{../../../}}

\setlength{\emergencystretch}{3cm}

\usepackage{subfiles}